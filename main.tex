% !TEX TS-program = pdflatex
% !TEX encoding = UTF-8 Unicode


%============================================================================%
%
%	DOCUMENMT DEFINITION
%
%============================================================================%

%we use report class for thesis design
\documentclass[11pt,titlepage,twoside,openright]{report}	


%----------------------------------------------------------------------------------------
%	ENCODING
%----------------------------------------------------------------------------------------

\usepackage[utf8]{inputenc}

%----------------------------------------------------------------------------------------
%	COLOR DEFINITIONS
%----------------------------------------------------------------------------------------

\usepackage{color}

%citation reference color
\definecolor{citationscol}{rgb}{0.1,0.1,0.5}

%urls color
\definecolor{urlscol}{rgb}{0.4,0.4,0.5}

%internal references color
\definecolor{linkscol}{rgb}{0.2,0.2,0.3}


%----------------------------------------------------------------------------------------
%	REFERENCES URLS CITATION STYLES AND SO ON
%----------------------------------------------------------------------------------------

%apply reference color definitions to hyperref package
\usepackage[colorlinks=true,citecolor=citationscol,urlcolor=urlscol,linkcolor=linkscol]{hyperref}
\usepackage{url}
\urlstyle{same}


%----------------------------------------------------------------------------------------
%	FONT DEFINITIONS
%----------------------------------------------------------------------------------------

% Better typography
\usepackage[protrusion=true,expansion=true]{microtype} 

 % Use the Palatino font
\usepackage{mathpazo}

 % Required for accented characters
\usepackage[T1]{fontenc}

% Change line spacing here, Palatino benefits from a slight increase by default
\linespread{1.05} 



%----------------------------------------------------------------------------------------
%	GEOMETRY  DEFINITIONS
%----------------------------------------------------------------------------------------

%define page styles using geometry
\usepackage{geometry}

 %define A4 Paper
\geometry{a4paper}			

% for example, change the margins to 2 inches all round
\geometry{margin=2.5cm} 	


%----------------------------------------------------------------------------------------
%	GRAPHICS  DEFINITIONS
%----------------------------------------------------------------------------------------

% Required for including pictures
\usepackage{graphicx}

 % make it possible to include more than one captioned figure/table in a single float 
\usepackage{subfig}


 % for much better looking tables
\usepackage{booktabs}	

% for better arrays (eg matrices) in maths		
\usepackage{array} 


%----------------------------------------------------------------------------------------
%	ENVIRONMENT  DEFINITIONS
%----------------------------------------------------------------------------------------


 % very flexible & customisable lists (eg. enumerate/itemize, etc.)				
\usepackage{paralist}				

% adds environment for commenting out blocks of text & for better verbatim
\usepackage{verbatim}				



%----------------------------------------------------------------------------------------
%	HEADER / FOOTER  DEFINITIONS
%----------------------------------------------------------------------------------------



 % This should be set AFTER setting up the page geometry
\usepackage{fancyhdr}

 % options: empty , plain , fancy
\pagestyle{fancy}

% use this for default header style
%\pagestyle{headings}	


%make chapter and section marks to lowercase
\renewcommand{\chaptermark}[1]{ \markboth{#1}{} }
\renewcommand{\sectionmark}[1]{ \markright{#1}{} }

 % customise the layout...
\renewcommand{\headrulewidth}{0pt}
\lhead{\thechapter - \thesection}\chead{}\rhead{\rightmark}
\lfoot{}\cfoot{\thepage}\rfoot{}


%----------------------------------------------------------------------------------------
%	SECTION TITLE APPEARANCE
%----------------------------------------------------------------------------------------

\usepackage{sectsty}

 % (See the fntguide.pdf for font help)
\allsectionsfont{\sffamily\mdseries\upshape}


%----------------------------------------------------------------------------------------
%	TABLE OF CONTENT APPEARANCE
%----------------------------------------------------------------------------------------

%\usepackage[nottoc,notlof,notlot]{tocbibind} % Put the bibliography in the ToC
\usepackage[titles,subfigure]{tocloft} % Alter the style of the Table of Contents
\renewcommand{\cftsecfont}{\rmfamily\mdseries\upshape}
\renewcommand{\cftsecpagefont}{\rmfamily\mdseries\upshape} % No bold!




\renewcommand{\abstractname}{Executive Summary} 		%switches abstract to executive summary





%============================================================================%
%
%	TITLE AND AUTHOR
%
%============================================================================%



\title{\textbf{The Master Thesis Title}}
\vspace{50pt} % Some vertical space between the title and author name
\author{\textsc{The Author}\\
	Dept of LaTex\\
	Tex University\\
	Textown, Texland\\
	\texttt{mail@texuni.com}}
\vspace{40pt} % Some vertical space between the title and author name
\date{\today}



%============================================================================%
%
%	INDEX AND GLOSSARY
%
%============================================================================%


\usepackage[style=long,nonumberlist,toc,xindy,acronym,nomain]{glossaries} % nomain, if you define glossaries in a file, and you use \include{INP-00-glossary}
%\loadglsentries[main]{glossary}
% or using \input:
\newacronym{lvm}{LVM}{Logical Volume Manager}
\newglossaryentry{Linux}
{
  name=Linux,
  description={is a generic term referring to the family of Unix-like
               computer operating systems that use the Linux kernel},
  plural=Linuces
}

\makeglossaries
\usepackage[xindy]{imakeidx}
\makeindex



%============================================================================%
%
%	BEGIN DOCUMENT
%
%============================================================================%


\begin{document}
\maketitle
\tableofcontents
\newpage


\begin{abstract}
Morbi tempor congue porta. Proin semper, leo vitae faucibus dictum, metus mauris lacinia lorem, ac congue leo felis eu turpis. Sed nec nunc pellentesque, gravida eros at, porttitor ipsum. Praesent consequat urna a lacus lobortis ultrices eget ac metus. In tempus hendrerit rhoncus. Mauris dignissim turpis id sollicitudin lacinia. Praesent libero tellus, fringilla nec ullamcorper at, ultrices id nulla. Phasellus placerat a tellus a malesuada.
\end{abstract}

\newpage

\chapter[Chapter Name]{Introduction}

Morbi tempor congue porta. Proin semper, leo vitae faucibus dictum, metus mauris lacinia lorem, ac congue leo felis eu turpis. Sed nec nunc pellentesque, gravida eros at, porttitor ipsum. Praesent consequat urna a lacus lobortis ultrices eget ac metus. In tempus hendrerit rhoncus. Mauris dignissim turpis id sollicitudin lacinia. Praesent libero tellus, fringilla nec ullamcorper at, ultrices id nulla. Phasellus placerat a tellus a malesuada.


\newpage

\section[Section Name]{First Section}

Morbi tempor congue porta. Proin semper, leo vitae faucibus dictum, metus mauris lacinia lorem, ac congue leo felis eu turpis. Sed nec nunc pellentesque, gravida eros at, porttitor ipsum. Praesent consequat urna a lacus lobortis ultrices eget ac metus. In tempus hendrerit rhoncus. Mauris dignissim turpis id sollicitudin lacinia. Praesent libero tellus, fringilla nec ullamcorper at, ultrices id nulla. Phasellus placerat a tellus a malesuada.

\begin{itemize}
	\item Morbi tempor congue porta. Proin semper, leo vitae faucibus dictum, metus mauris lacinia lorem, ac congue leo felis eu turpis. 
	\item Morbi tempor congue porta. Proin semper, leo vitae faucibus dictum, metus mauris lacinia lorem, ac congue leo felis eu turpis. 
	\item Morbi tempor congue porta. Proin semper, leo vitae faucibus dictum, metus mauris lacinia lorem, ac congue leo felis eu turpis. 
	\item Morbi tempor congue porta. Proin semper, leo vitae faucibus dictum, metus mauris lacinia lorem, ac congue leo felis eu turpis. 
\end{itemize}

\subsection[Subsection Name]{A subsection}

Morbi tempor congue porta. Proin semper, leo vitae faucibus dictum, metus mauris lacinia lorem, ac congue leo felis eu turpis. Sed nec nunc pellentesque, gravida eros at, porttitor ipsum. Praesent consequat urna a lacus lobortis ultrices eget ac metus. In tempus hendrerit rhoncus. Mauris dignissim turpis id sollicitudin lacinia. Praesent libero tellus, fringilla nec ullamcorper at, ultrices id nulla. Phasellus placerat a tellus a malesuada.

More text. More text. More text. \index{more here} More text. More text. More text. More text. More text. More text. 



%============================================================================%
%
%	APPENDIX
%
%============================================================================%

\cleardoublepage
\phantomsection %hyperref package support
\addcontentsline{toc}{chapter}{Appendix}
{\textbf{\LARGE{Appendix}}}\\ %headline

Some Appendix Text. \index{Appendix} Some Appendix Text. Some Appendix Text.


%============================================================================%
%
%	BIBLIOGRAPHY
%
%============================================================================%
%useful: https://en.wikibooks.org/wiki/LaTeX/Bibliography_Management

%------------ linking bib to table of content ------------------%
\cleardoublepage
\phantomsection %hyperref package support
\addcontentsline{toc}{chapter}{Bibliography}

%------------ the bibliography ------------------%

{\textbf{\LARGE{Bibliography}}}\\	%headline
\nocite{*} % Show all Bib-entries (DEBUG)
%\bibliographystyle{plainnat}
\bibliographystyle{plain}
\bibliography{literature.bib}


%============================================================================%
%
%	INDEX AND GLOSSARY
%
%============================================================================%

\cleardoublepage
\phantomsection %hyperref package support
\addcontentsline{toc}{chapter}{Index}
{\textbf{\LARGE{Index}}}\\ %headline
\printindex
%\printglossary[title=List of Terms,toctitle=Terms and abbreviations]
\printglossaries

\end{document}
